\documentclass[]{article}
\usepackage{lmodern}
\usepackage{amssymb,amsmath}
\usepackage{ifxetex,ifluatex}
\usepackage{fixltx2e} % provides \textsubscript
\ifnum 0\ifxetex 1\fi\ifluatex 1\fi=0 % if pdftex
  \usepackage[T1]{fontenc}
  \usepackage[utf8]{inputenc}
\else % if luatex or xelatex
  \ifxetex
    \usepackage{mathspec}
    \usepackage{xltxtra,xunicode}
  \else
    \usepackage{fontspec}
  \fi
  \defaultfontfeatures{Mapping=tex-text,Scale=MatchLowercase}
  \newcommand{\euro}{€}
\fi
% use upquote if available, for straight quotes in verbatim environments
\IfFileExists{upquote.sty}{\usepackage{upquote}}{}
% use microtype if available
\IfFileExists{microtype.sty}{%
\usepackage{microtype}
\UseMicrotypeSet[protrusion]{basicmath} % disable protrusion for tt fonts
}{}
\usepackage[margin=1in]{geometry}
\ifxetex
  \usepackage[setpagesize=false, % page size defined by xetex
              unicode=false, % unicode breaks when used with xetex
              xetex]{hyperref}
\else
  \usepackage[unicode=true]{hyperref}
\fi
\hypersetup{breaklinks=true,
            bookmarks=true,
            pdfauthor={Andrea Fernández},
            pdftitle={Proyecto 3: Orquestación},
            colorlinks=true,
            citecolor=blue,
            urlcolor=blue,
            linkcolor=magenta,
            pdfborder={0 0 0}}
\urlstyle{same}  % don't use monospace font for urls
\setlength{\parindent}{0pt}
\setlength{\parskip}{6pt plus 2pt minus 1pt}
\setlength{\emergencystretch}{3em}  % prevent overfull lines
\setcounter{secnumdepth}{5}

%%% Use protect on footnotes to avoid problems with footnotes in titles
\let\rmarkdownfootnote\footnote%
\def\footnote{\protect\rmarkdownfootnote}

%%% Change title format to be more compact
\usepackage{titling}
\setlength{\droptitle}{-2em}
  \title{Proyecto 3: Orquestación}
  \pretitle{\vspace{\droptitle}\centering\huge}
  \posttitle{\par}
  \author{Andrea Fernández}
  \preauthor{\centering\large\emph}
  \postauthor{\par}
  \predate{\centering\large\emph}
  \postdate{\par}
  \date{30/05/2015}


\usepackage{float}
\usepackage{morefloats}
\usepackage[spanish]{babel}
\usepackage{graphicx}
\usepackage{tcolorbox}
\usepackage{rotating}
\usepackage{longtable}
\usepackage{colortbl}
%\usepackage{natbib}
%\newenvironment{scaleb}{ \scalebox{0.4}{} {} }
%\newenvironment{scaleb}{ \tiny{} }
% biber
\usepackage[autostyle]{csquotes}

\usepackage[
    backend=biber,
    style=authoryear-icomp,
    sortlocale=de_DE,
    natbib=true,
    url=false,
    doi=true,
    eprint=false
]{biblatex}
\addbibresource{bibliografia.bib}

\usepackage[]{hyperref}
\hypersetup{
% Turn on this if you prefer to have links colored instead of marked with squares
colorlinks = true,
linkcolor = black,
urlcolor = blue,
citecolor = black,
% pdfpagemode = UseNone
}

\renewcommand\figurename{Figura}
\renewcommand\tablename{Tabla}

\newenvironment{myexampleblock}[1]{%
    \tcolorbox[beamer,%
    noparskip,breakable,
    colback=LightGreen,colframe=DarkGreen,%
    colbacklower=LimeGreen!75!LightGreen,%
    title=#1]}%
    {\endtcolorbox}

\newenvironment{myalertblock}[1]{%
    \tcolorbox[beamer,%
    noparskip,breakable,
    colback=LightCoral,colframe=DarkRed,%
    colbacklower=Tomato!75!LightCoral,%
    title=#1]}%
    {\endtcolorbox}

\newenvironment{myblock}[1]{%
    \tcolorbox[beamer,%
    noparskip,breakable,
    colback=LightBlue,colframe=DarkBlue,%
    colbacklower=DarkBlue!75!LightBlue,%
    title=#1]}%
    {\endtcolorbox}


\begin{document}

\maketitle


{
\hypersetup{linkcolor=black}
\setcounter{tocdepth}{3}
\tableofcontents
}
\section{Introducción}\label{introduccion}

\section{Objetivo}\label{objetivo}

Implementar el flujo:

\textbf{GDELT} - Obtención: - Crawler - Identificar si hay nueva
información - Bajar a zip - Guardar en disco - Descomprimir - Limpieza:
- Revisar columnas - Quitar columnas no requeridas - Estructurar las
columnas en un orden apropiado - Realizar proceso de normalización de
datos (e.g.~fechas a UTC) - Enviar a un archivo de texto - Manipulación:
- Detectar que se escribió un archivo de texto y triggerear la subida al
HDFS - Realizar un proceso de analítica que actualice la información -
Tener una base de datos para shiny actualizada

\section{Herramientas a utilizar}\label{herramientas-a-utilizar}

\begin{enumerate}
\def\labelenumi{\arabic{enumi}.}
\itemsep1pt\parskip0pt\parsep0pt
\item
  Sistema de carpetas / configuración / docker
\item
  Flume
\item
  Luigi
\item
  Pyspark
\item
  Sqoop
\item
  Hive/Impala
\end{enumerate}

\section{Configuración}\label{configuracion}

\section{Apache Flume}\label{apache-flume}

\subsection{¿Qué es?}\label{que-es}

\subsection{¿Para qué se utiliza?}\label{para-que-se-utiliza}

\subsection{Ventajas y desventajas}\label{ventajas-y-desventajas}

\subsection{Implementación}\label{implementacion}

\begin{verbatim}
esto es codigo
\end{verbatim}

\section{pyspark}\label{pyspark}

\subsection{¿Qué es?}\label{que-es-1}

\subsection{¿Para qué se utiliza?}\label{para-que-se-utiliza-1}

\subsection{Ventajas y desventajas}\label{ventajas-y-desventajas-1}

\subsection{Implementación}\label{implementacion-1}

\section{Orquestación vía Luigi}\label{orquestacion-via-luigi}

(Vease el ejemplo de Karau, Konwinski, Wendell, and Zaharia, 2015,
p.~345) Karau, Konwinski, Wendell, and Zaharia (Vease el ejemplo de
2015, p.~345)

\subsection{¿Qué es?}\label{que-es-2}

\subsection{¿Para qué se utiliza?}\label{para-que-se-utiliza-2}

\subsection{Ventajas y desventajas}\label{ventajas-y-desventajas-2}

\subsection{Implementación}\label{implementacion-2}

\section{Hive}\label{hive}

\subsection{¿Qué es?}\label{que-es-3}

\subsection{¿Para qué se utiliza?}\label{para-que-se-utiliza-3}

\subsection{Ventajas y desventajas}\label{ventajas-y-desventajas-3}

\subsection{Implementación}\label{implementacion-3}

\section{Conclusiones}\label{conclusiones}

\section{Bibliografía}\label{bibliografia}

{[}1{]} H. Karau, A. Konwinski, P. Wendell and M. Zaharia.
\emph{Learning Spark: Lightning-Fast Big Data Analysis}. " O'Reilly
Media, Inc.``, 2015.

\end{document}
